%%%%%%%%%%%%%%%%%%%%%%%%%%%%%%%%%%%%%%%%%%%%%%%%%%%%%%%%%%%%%%%%%%%%%%%%
\documentclass[12pt]{article}
\usepackage{amssymb,amsthm}
\usepackage{amsmath,amssymb,CJK}
\usepackage{graphicx}
\usepackage{subfigure}
\usepackage{listings}
\usepackage{enumerate}

\openup 7pt\pagestyle{plain} \topmargin -40pt \textwidth
14.5cm\textheight 21.5cm
\parskip .09 truein
\baselineskip 4pt\lineskip 4pt \setcounter{page}{1}
\def\a{\alpha}\def\b{\beta}\def\d{\delta}\def\D{\Delta}\def\fs{\footnotesize}
\def\g{\gamma}
\def\G{\Gamma}\def\l{\lambda}\def\L{\Lambda}\def\o{\omiga}\def\p{\psi}
\def\se{\subseteq}\def\seq{\subseteq}\def\Si{\Sigma}\def\si{\sigma}\def\vp{\varphi}\def\es{\varepsilon}
\def\sc{\scriptstyle}\def\ssc{\scriptscriptstyle}\def\dis{\displaystyle}
\def\cl{\centerline}\def\ll{\leftline}\def\rl{\rightline}\def\nl{\newline}
\def\ol{\overline}\def\ul{\underline}\def\wt{\widetilde}\def\wh{\widehat}
\def\rar{\rightarrow}\def\Rar{\Rightarrow}\def\lar{\leftarrow}
\def\Lar{\Leftarrow}\def\Rla{\rightleftarrow}\def\bs{\backslash}
\def\ra{\rangle}\def\la{\langle}\def\hs{\hspace*}\def\rb{\raisebox}
\def\ni{\noindent}\def\hi{\hangindent}\def\ha{\hangafter}
\def\vs{\vspace*}
\def\hom#1{\mbox{\rm Hom($#1,#1$)}}
\def\thebeg{\vskip8pt\par\ni}
\def\theend{\vskip5pt\par}
\def\chabeg{\pagebreak\par}
\def\chaend{\vskip20pt\par}
\def\secbeg{\vskip15pt\par}
\def\secend{\vskip10pt\par}
\def\exebeg{\vskip10pt}
\def\exeend{\vskip6pt}
\def\undot#1{\mbox{$\mbox{#1}\hs{-1.5ex}_{_{\dis\centerdot}}\,\,$}}
\def\qed{\hfill\mbox{$\Box$}}
\def\C{\mathbb{C}}
\def\Q{\mathbb{Q}}
\def\ii{\mbox{\,{\bf i}\,}}
\def\jj{\mbox{\,{\bf j}\,}}
\def\AA{{\cal A}}
\def\BB{{\cal B}}
\def\DD{{\cal D}}
\def\EE{{\mbox{\bf 1}}}
\def\OO{{\mbox{\bf 0}}}
\def\kk{{\mbox{\bf k}}}
\def\R{\mathbb{R}}
\def\F{\mathcal{F}{\ssc\,}}
%\def\similar{\rb{-4pt}{\mbox{\,\~\,}}}
\def\similar{\backsim}
\def\Llra{\Longleftrightarrow}
\def\Lra{\Longrightarrow}
\def\Lla{\Longleftarrow}
\def\mat#1#2{\mbox{$\left(\begin{array}{#1}#2\end{array}\right)$}}
\def\det#1#2{\mbox{$\left|\begin{array}{#1}#2\end{array}\right|$}}
\def\eset{\emptyset}
\parindent=5ex
\setlength{\parindent}{0pt}
\setlength{\parskip}{1ex plus 0.5ex minus 0.2ex}
\newtheorem{Example}{\text{例}}
\begin{CJK*}{UTF8}{gbsn}

\date{}
\title{Homework3}
\author{Qinglin Li, 5110309074}
\begin{document}
\maketitle
\section*{Problem 1}
\begin{enumerate}
\item
Let $X$ be a subset of $V$. $\forall v\in V$, with probability $p$, $v \in X$\\
$$E[|X|]=np$$
Let $Y$ be the set of all vertices in $V-X$ that have no neighbours in $X$\\
$$E[|Y|]=n(1-p)^{d+1}$$
$X\cup Y$ is a dominating set\\
\begin{align*}
E[|X\cup Y|]&\leq E[|X|]+E[|Y|]=np+n(1-p)^{d+1}\\
&= n\left[p+(1-p)^{d+1}\right] \leq n\left[p+e^{-p(d+1)}\right]
\end{align*}
let $p=\dfrac{\ln(d+1)}{d+1}$
$$E[|X\cup Y|]\leq n\left[\dfrac{\ln(d+1)}{d+1}+\dfrac{1}{d+1	}\right]=\dfrac{n(1+\ln(d+1))}{d+1}$$
\item
Let $X$ be a subset of $V$. $\forall v\in V$, with probability $p$, $v \in X$\\
Let  $A_i$ be the event that neither the i-th vertex nor its neighbors are in $X$\\
By Lovasz local lemma, if $e(1-p)^{d+1}n\leq 1$, with nonzero probability none of these events happens .\\
$\because e(1-p)^{d+1}n\leq e^{1-p(d+1)}n$\\
$\therefore  e^{1-p(d+1)}n\leq 1 \Rightarrow e(1-p)^{d+1}n\leq 1$\\
$\therefore p\geq \dfrac{\ln n+1}{d+1}$\\
$\therefore$ the bound is $n\dfrac{\ln n+1}{d+1}$\\
So the bound is worse.\\
\end{enumerate}

\section*{Problem 2}
Suppose $G(V,E)$ dosen't contain $H$\\
Covering $K_n$ with $k$ graphs isomorph of $G$ is corresponding to color $K_n$ with $k$ colors.\\
If an edge is covered more than once, let it be of the first color.\\
Since $H$ is not a subgraph of the $k$ graphs, if every edge is covered,  a k-coloring meeting the conditions exists.\\
let 
$
X_i = \left\{
	\begin{aligned}
	&1~~\text{the i-th edge is not covered}\\
	&0~~\text{otherwise}\\
	\end{aligned}
\right.
$\\
let $X=\sum X_i$
$$E[X]=\sum E[X_i]=\binom{n}{2}\left(1-\dfrac{m}{\binom{n}{2}}\right)^k\leq \dfrac{n^2}{2}e^{\frac{-2mk}{n^2}}$$
let $k=\dfrac{n^2\ln n}{m}, E[X]<1$\\
 So there is an edge k-coloring for $K_n$ that $K_n$ contains no monochromatic $H$. 
\section*{Problem 3}
\textbf{Lemma:}\\
Let $G(V,E)$ be a graph with degree at most $1$ for each vertex and have $2n$ vertices. \\
Let $V=V_1\cup V_2 \cup \cdots \cup V_n$ be a partition s.t. $\forall V_i, |V_i|=2$\\
Then there exists an independent set of $G$ containing precisely one vertex from each $V_i$\\

The independent can be find in the following way:
\begin{enumerate}
	\item Arbitrarily choose a vertex $v_1$ for an arbitrary set $V_1$
	\item For each $i$, suppose $v \in V_i$ is choosen.\\
	 If $\exists w~s.t.~ (v,w)\in E\wedge \forall j\leq i, w\not\in V_j$, let the set containing $w$ be $V_{i+1}$. Else let any remaining set be $V_{i+1}$. \\
	 Find $v_{i+1}\in V_{i+1}~s.t.~(v_i, v_{i+1})\not\in E$\\
\end{enumerate}	 
Let $G=(V, E)$ be a cycle of length $4n$. Reorder the vertices of $G$ such that $E=\{\left(i,(i+1)\mathrm{mod~n}\right)|i\leq 4n\}$\\
Let $G'=(V,E')$ with $E'=\{(2i-1,2i)|i\leq 2n\}\subset E$. \\
Each vertex in $G'$ has exact degree $1$\\
Split $V_i$ of $G$ into $2$ sets $V_{i1}$ and $V_{i2}$\\
So there must be an independent set $S$ of $G'$ contains exact $1$ vertex of each $V_{ij}~(j\in\{1,2\})$\\
Then an independent set $S'$ of $V\setminus S$ with $V_i\setminus S$ can be found\\
And $S'$ is obviously an independent set of $G$ containing precisely one vertex from each $V_i$

\section*{Problem 4}
Let $V\subseteq U$, each element in $U$ present in $V$ with probability $p$
\begin{align*}
&q\triangleq\Pr(V \text{is isolating set})\\
=&\Pr(V\cup S=\emptyset)\left[1-\Pr(T_i\cap V=\emptyset\vee\cdots\vee T_m\cap V=\emptyset)\right]\\
\geq & (1-p)^{|S|}\left[1-m(1-p)^{|S|}\right]
\end{align*}
Let $p=\dfrac{\ln 2m}{|S|}$, $\Pr(V \text{is isolating set})\geq\dfrac{1}{2m}\cdot \dfrac{1}{2}=\dfrac{1}{4m}$\\
Let $N=|\F|$\\
 $\forall$ safe set instance $I$,let 
$
X_I=\left\{
	\begin{aligned}
	1& ~~~\text{no isolating set in } \F\\
	0& ~~~\text{at least one isolating set in } \F\\
	\end{aligned}
\right.
$\\
Let $X=\sum_I X_I$, \# of distinct $\displaystyle I = \binom{n}{\left|S\right|}\binom{n-|S|}{|S|}^m\leq 2^{n(m+1)}$ 
$$E[X]=\sum_I E[X_i]\leq 2^{n(m+1)}(1-q)^N$$
$$\Pr(X=0)=1-\Pr(X\geq 1)\geq 1-E[X]$$
Let $N>4mn(m+1)$
 $$\Pr(X=0)>2^{n(m+1)}\left(1-\dfrac{1}{4m}\right)^{4mn(m+1)}\approx\left(\dfrac{2}{e}\right)^{n(m+1)}>0$$

\section*{Problem 5}
Let $Y=\left|\dfrac{X}{\delta}\right|$ \\
$$\dfrac{E\left[e^{t|X|}\right]}{e^{t\delta}}=\dfrac{E\left[e^{t\delta Z}\right]}{e^{t\delta}}=\dfrac{E\left[\sum_{i=0}^{\infty}\frac{(t\delta)^i}{i!}Y^i\right]}{e^{t\delta}}=\sum_{i=0}^{\infty}\frac{(t\delta)^i}{i!e^{t\delta}}E[Y^i]$$
$\because\displaystyle \sum_{i=0}^{\infty}\frac{(t\delta)^i}{i!e^{t\delta}}=1$\\
$\therefore $ let $k=\mathrm{argmin_i} E[Y^i], E[Y^k]=\dfrac{E\left[|X|^k\right]}{\delta^k}\leq \dfrac{E\left[e^{t|X|}\right]}{e^{t\delta}}$

We prefer Chernoff bound because it's easy to use and calculate.\\
\end{CJK*}
\end{document}